\documentclass[upright, contnum]{umemoria}
\depto{DEPARTAMENTO DE INGENIER�A EL�CTRICA}
\author{FRANCO ANDREAS CUROTTO MOLINA}
\title{GRAPHSLAM ALGORITHM IMPLEMENTATION FOR sOLVING THE PROBLEM OF SLAM}
\auspicio{}
\date{DICIEMBRE 2015}
\guia{MARTIN ADAMS}
\carrera{INGENIERO CIVIL EL�CTRICO}
\memoria{MEMORIA PARA OPTAR AL T�TULO DE}
\comision{\ Marcos Eduardo Orchard Concha}{\ Jorge Felipe Silva S�nchez} %{\ }

\usepackage{lipsum}

\usepackage[latin1]{inputenc}
\usepackage[T1]{fontenc}

\usepackage{amsmath} % mathematical equations
\usepackage{amssymb} % add maths symbols
\usepackage{mathtools} % more maths symbols

\usepackage{graphicx} % add figures
\usepackage{caption} % caption customization
\usepackage{subcaption} % subfigures caption customization
%\usepackage[a4paper]{geometry} % page layout modification

\usepackage{algorithm}
\usepackage{algpseudocode}

\usepackage[disable]{todonotes}

%\newcommand{\mem}{Memoria}
\newcommand{\Nn}{\mathcal{N}}
\newcommand{\Cc}{\mathcal{C}}
\newcommand{\bs}{\boldsymbol}
\newcommand{\defi}{\vcentcolon=}
\renewcommand{\it}{\textit}

\newcommand{\estWidth}{0.8}
\newcommand{\errorWidth}{0.5}


\begin{document}

\frontmatter
\maketitle

\begin{abstract}
SLAM (Simultaneous Localization and Mapping) is the problem of concurrently estimate a robot (or other agent) position and the map the robot is generating of the environment. It is consider to be a core concept in mobile robotics, and fundamental to achieve truly autonomous systems. 

Among the many solution that has been proposed for solving SLAM, graph based approaches have gain significant interest from researches in the recent years. These solutions presents various advantages, such as, the capability to handle larges amount of data, and to retrieve the whole robot trajectory, rather than just the last position. A particular implementation of this approach is the GraphSLAM algorithm, first presented Thrun and Montemerlo in 2006.

In the present work a GraphSLAM algorithm is implemented for solving the SLAM problem in a two dimensional scenario. The main objective of this work is to provide of a widely accepted SLAM solution for benchmark comparison for newer SLAM algorithms. The implementation uses the g$^2$o framework as a tool for nonlinear least square optimization.  The algorithm utilize a sparse matrix to store the robot uncertainty in an efficient manner. 

The GraphSLAM implementation is able to solve SLAM with known and unknown data association. This means that even with the robot is incapable to distinguish two measurements belonging to the same landmark, it can assert this its equivalence by computing the likelihood of the landmarks being the same. The algorithm also uses a kernel method for robust estimation against outliers. In order to improve the algorithm computational time, several strategies were designed to efficiently test the association between landmarks, and run the optimizations. 

The final implementation was tested with simulated and real data, in the case of known and unknown data association. It worked successfully in all the tasted cases, being able to retrieved the robot path and the environment map with a low error. The main advantage of the algorithm are the high precision and the available number of parameters to accommodate it to a wide range of scenarios. The mayor drawbacks are the algorithm computational time for large dataset, and the inability to deal with false alarms. 

Finally, future work is given in order to improve the implementation current performance. In particular, modifications are suggested to increase convergence speed, and a method for dealing with false positives is proposed.   
\end{abstract}

\begin{dedicatoria} % opcional
Una dedicatoria corta. Por ejemplo, \emph{A los creadores de U-Campus}
\end{dedicatoria}

\begin{thanks} % opcional
%\lipsum[1-2]
\end{thanks}
\cleardoublepage

\tableofcontents
\listoftables % opcional
\listoffigures % opcional

\mainmatter

%\input{intro.tex}
\listoftodos

\chapter{Introduction}

%This work is immerse in the context of Mobile Robotics. Mobile
%Robotics is the branch of engineering that study machine that can move in an
%environment, that is, they can change their location over time. Traditionally,
%robots were usually used to do a simple, repetitive task in a fixed location, such
%as robots found in assembly lines. In contrast, mobile robots are more versatile
%and able to do a wider variety of tasks, but at the cost of needing more complex
%models to study and control them.
%
%Mobile robots frequently have to work in unknown environments, with high
%uncertainly. Researches have found that a good way to deal with this uncertainly
%is to treat the involved variables in the problem as random variables. This way
%the field of Probabilistic Robotics was born.
%
%Two of the main problems to be solved in mobile robotics are localization and
%mapping. Localization means to find an estimate of the location of a robot that is
%moving on a scene. Mapping is the problem of constructing a map of an unknown
%environment. When the agent in charge of constructing the map is a robot moving
%in the same environment, both localization and mapping must be solve at the same
%time. In Robotics this problem is called Simultaneous Localization and Mapping
%(SLAM for short). SLAM is a widely studied problem in the academy, and wide
%variety of solutions exists, all with their own advantages and disadvantages.
%One particular solution for solving SLAM is GraphSLAM. GraphSLAM was
%first developed by Sebastian Thrun and Michael Montemerlo in 2006, and to date
%is considered one of the most robust solutions, in addition of been simple and
%of relative low complexity. GraphSLAM represents the necessary information re-
%garding the robot and the map as nodes of a graph. This provides the algorithm
%with advantages, such as the ability to store the information efficiently, in sparse
%matrices.
%
%The main objective of this work is to implement the GraphSLAM
%algorithm. To the author knowledge here is no freely available source code of
%GraphSLAM, and its implementation is invaluable as a benchmark comparison for
%newer SLAM algorithms.
%
%The contribution of this work is to provide a fully functional SLAM
%algorithm, which could be used for the navigation of robots in real world scenarios,
%and for the realization of comparative analysis with other SLAM algorithms.

The study of robotic systems is fundamental to achieve the increasingly demanding goal of automating the processes that occur in every aspect of our lives. Autonomous systems are becoming more ubiquitous by the day, while historically they were first used in manufacturing companies and industrial processes, now they have found applications in areas like farming, mining, transportation, security, medicine, household maintenance, space exploration, military uses and much more.

In particular, mobile robotics is the study of mechanical agent that move in an environment. Their ability to change its positions make mobile robots capable of doing a much wider range of tasks than stationary robots. However, their motion capability comes with an essential problem: as the robot moves, it must compute its new positions to continue working properly. In robotics, the problem of estimating the robot current position is called \textit{localization}. Sensors are used to gather information about the robot location, however, any kind of sensor is contaminated with noise, so robot position can't be retrieved with absolute certainty, but it must be estimated by means of probabilistic methods.  

%Broadly, there are two types of sensors used for localization, motion sensors to measure the displacement of the robot in certain timestep, and position sensors that measure its relative position to certain portions of the environment.

On the other hand, if the environment in which the robot is immerse is also unknown, it must be estimated alongside with the position. The problem of getting an estimate of the robot environment is called \textit{mapping}. When both localization and mapping must be solved concurrently, the problem is called \textit{Simultaneous Localization and Mapping} (SLAM). SLAM is considered somewhat as the ``Holy Grail'' of mobile robotics, as knowing both, robot location and the environment, are crucial for every robot to work properly. The subfield of robotics that studies the probabilistic method and algorithms to solve problems such as SLAM is usually called Probabilistic Robotics.

Currently, one of the most widely used algorithms to solve SLAM is GraphSLAM. GraphSLAM was developed by Sebastian Thrun and Michael Montemerlo~\cite{graphslam}, and to date is considered one of the most robust solutions, in addition of been simple and of relative low complexity. GraphSLAM represents the necessary information regarding the robot and the map as nodes of a graph. The graph can then be converted to a sparse matrix, for which efficients algorithms exist to compute the robot's path and the map. 


\section{General Objectives}

The main objective of this work is to implement a offline version of the GraphSLAM algorithm for solving 2D SLAM. The implementation should be able to handle known and unknown data association, and be robust to non-Gaussian noise, outliers and false alarms\todo{review}.

The contribution of this work is to provide a fully functional SLAM algorithm, which could be used for the navigation of robots in real world scenarios, and as a benchmark comparison for newer SLAM algorithms.

\section{Specific Objectives}

\begin{enumerate}
    \item Understand and document the g$^2$o framework. g$^2$o is a framework written in \texttt{C++} for graph optimization.\todo{review}
    \item Implement a data association algorithm to handle landmarks of unknown correspondence.
    \item Test the implementation with simulated and real data. 
\end{enumerate}

\section{Document Structure}

The remainder of the document is organized as follows. In chapter~\ref{chap:antecedents} the basic concepts of Probabilistic Robotics and SLAM, as well as the theoretical framework of the GraphSLAM algorithm is presented. In chapter~\ref{chap:implementation} the implemented GraphSLAM algorithm is explained in detail. In chapter~\ref{chap:results} the results of the implementations are shown for various scenarios, and a parameter analysis is made. Finally in Conclusi\'on the results are discussed, and possible future work is suggested. 

\input{cap2.tex}
\chapter{Methodology and Implementation}
\label{chap:implementation}
 
In this chapter the implementation of the GraphSLAM algorithm is presented and explained. The implementation created is capable of solving the SLAM problem in an offline manner for a 2D scenario, in the cases of known and unknown data association.

The g$^2$o framework used in this work provides of a least squares solver for the optimization of equation~\eqref{eq:minimization}, as well as a protocol to define a graph in the SLAM problem. g$^2$o is well optimized and it has several options for the solver, so the known correspondence version of the GraphSLAM algorithm is relative straightforward to implement.

However, g$^2$o doesn't provide a way to handle unknown data association, so the main goal of this work is to implement a method for solving the correspondence problem in an efficient manner.

\section{The g$^2$o Protocol}

The first step to implementing the GraphSLAM algorithm is to define a protocol to store and interpret the data on a graph. g$^2$o already provides such protocol. The stored data comes in two types: data from nodes, and data from edges. 

In SLAM context, nodes itself can be of two types, pose nodes and landmark nodes. Pose nodes represent the pose of the robot, in the 2D case they consists in 3 variables: robot's $x$ and $y$ position, and robot's bearing $\theta$. In g$^2$o a pose node is denoted with the keyword \texttt{VERTEX\_SE2}\footnote{Vertex is synonymous of node. SE2 is the Non-Euclidean space that consists of two spatial dimensions and a angular dimension.}. Landmark nodes represent the 2D position ($x$ and $y$) of a landmark. They are denoted with the keyword \texttt{VERTEX\_XY}.

Edges represents either robot's odometry (data of robot's change in position), or robot's measurements of landmarks. Odometry is measured as the difference between robot's pose at two consecutive timesteps: $(\Delta x, \Delta y, \Delta \theta)$. On the other hand, robot measurement are given as the $x$ and $y$ distance to the landmark relative to the robot reference frame. Figure~\ref{fig:protocol} illustrates the odometry an measurement of a robot. Keywords \texttt{EDGE\_SE2} and \texttt{EDGE\_SE2\_XY} are use to denote odometry and measurement edges respectively.

\begin{figure}[htbp!]
    \centering
    \includegraphics[width=0.7\textwidth]{tikz/protocol.pdf}
    \caption{Ilustration of odometry and measurements values in g$^2$o.}
    \label{fig:protocol}
\end{figure}

For GraphSLAM to work correctly, one must also provide to the algorithm of the uncertainty of odometry and measurements. These correspond to the covariance matrices $\bs{R}_k$ and $\bs{Q}_k$ from motion~\eqref{eq:motion-model} and measurement~\eqref{eq:measurement-model} model respectively. Actually g$^2$o works with the inverse of the covariance matrix, known as the information matrix, however, these representation are equivalent in terms of the knowledge of the system. Since covariance matrices (and their inverse) are symmetrical, only the upper diagonal elements are needed. In this work it is assumed that the model's uncertainly are time independent, i.e., all nodes have the same values for the information matrix. The notation of each element of the matrices is shown as follows:

\begin{equation}
    \bs{R}^{-1}_k = \begin{pmatrix}
    ipxx & ipxy & ipxt\\
    ipxy & ipyy & ipyt\\
    ipxt & ipyt & iptt
    \end{pmatrix} \;\;
    \bs{Q}^{-1}_k = \begin{pmatrix}
    ilxx & ilxy\\
    ilxy & ilyy
    \end{pmatrix} 
    \label{eq:info-matrices}
\end{equation}

Finally, nodes must be indexed so they can be distinguishable form other nodes, indicated by an integer \texttt{id}. Ids are used in edges to indicate the two nodes that the edge is connecting.

Table~\ref{tab:protocol} summarize the g$^2$o notation to represent nodes and edges:

\begin{table}[htbp!]
    \centering
    \begin{tabular}{|c|c|}
        \hline
        Graph element & Notation\\
        \hline
        Pose node & \texttt{VERTEX\_SE2 id x y t}\\
        Landmark node & \texttt{VERTEX\_SE2\_XY id x y}\\
        Odometry edge & \texttt{EDGE\_SE2 id1 id2 dx dy dt ipxx ipxy ipxt ipyy ipyt ipyy}\\
        Measurement edge & \texttt{EDGE\_SE2\_XY id1 id2 dx dy ilxx ilxy ilyy}\\
        \hline
    \end{tabular}
    \caption{g$^2$o protocol for node and edge definition.}
    \label{tab:protocol}
\end{table}

In a SLAM problem one usually have only information about robot odometry and measurement, so only edges data is known. All the data can be written in a plain text, that can be latter uploaded to the g$^2$o framework.

\section{The Known Correspondence Case}

Using g$^2$o solving the known correspondence case is just a matter of loading the data to the framework, set the desired parameters, and running the solver. 

The parameters that can be set in the solver includes: 

\begin{enumerate}
    \item The sparse solver for the inversion in~\eqref{eq:linear-system}: Cholesky solver, PCG solver
    \item The optimization algorithm for solving~\eqref{eq:linearized}-\eqref{eq:update}: Gauss-Newton, Levenberg-Marquardt
    \item The number of iteration for the algorithm to stop.
\end{enumerate}

For the sparse solver, g$^2$o uses third-parties libraries from which the user can choose: CHOLMOD, CSparse\footnote{CHOLMOD and CSparse can be found in \url{http://faculty.cse.tamu.edu/davis/suitesparse.html}}, Eigen\footnote{\url{http://eigen.tuxfamily.org/index.php?title=Main_Page}}. 

This work provides of a Python script to easily set the parameters of the framework and run the solver. It also provides of a 2D simulator that generates a robot path, landmarks and measurements. It's a modified version of the g$^2$o simulator that allows the user to set the information parameters from matrices~\eqref{eq:info-matrices} in the simulations. This make possible to test the GraphSLAM algorithm for different noise levels. The simulation also allows to compare the results with the ground truth (the true values of the robot path and landmarks generated by the simulator). 

The Python script is also able to generate an initial guess of the estimate using robot odometry and the first measurement of each landmark. The initial guess is necessary for the optimization algorithm to work. Figure~\ref{fig:simulation} shows an example of the ground truth and the initial guess of a SLAM simulation.

\begin{figure}[htbp!]
    \centering
    \includegraphics[width=0.8\textwidth]{imagenes/guess_op_100_oa_100_lp_100_ds_100.pdf}
    \caption{Example of an SLAM simulation. The light blue line is ground truth path, and dark red circles are ground truth landmark. The blue line is odometry path, red cross are initial guess landmarks.}
    \label{fig:simulation}
\end{figure}

The pseudocode for the known correspondence case is shown in Algorithm~\ref{alg:known-correspondence}.

\begin{algorithm}
    \caption{GraphSLAM Known Correspondence}
    \label{alg:known-correspondence}
    \begin{algorithmic}[1]
        \Require optimizer, data
        \State optimizer.setParameters(parameters)
        \State optimizer.loadData(data)
        \State optimizer.genInitialGuess()
        \State optimizer.solve(numberIterations)
        \State optimizer.writeData()
    \end{algorithmic}
\end{algorithm}

Optionally g$^2$o can use robust kernels to deal with outliers. An outlier is a corrupt measurement that doesn't follows the distribution assumed for the model. They are usually generated by sensors malfunctions and tend to have extreme values, far away form the expected measurement. 

From equation~\eqref{eq:simplified}, it can be seen that each error function has a quadratic influence in function $F(\bs{y})$. This means that a single outlier can significantly degrade the construction of $F$, and thus the result of the estimation. To mitigate this problem a robust kernel function can be applied to each error term $\bs{e}_{ij}(\bs{y})$ in~\eqref{eq:simplified}, so that high values of $\bs{e}_{ij}$ has reduced effect in $F$. Robust kernels included in g$^2$o are: Cauchy, DCS, Fair, GemanMcClure, Huber, PseudoHuber, Saturated, Tukey, Welsch. Most robust kernels must also specify the kernel's width, that is the point on the function in which the kernel effect start. Figure~\ref{fig:kernels} shows the plots of different kernels in g$^2$o. 

\begin{figure}[htbp!]
    \centering
    \includegraphics[width=0.8\textwidth]{imagenes/kernels.pdf}
    \caption{Plots of different robust kernels with equals width, compared with the quadratic function.}
    \label{fig:kernels}
\end{figure} 

In this work a Huber kernel of width 1 is used.

\section{The Unknown Correspondence Case}

In the unknown correspondence case there is no information of which landmark generates each measurement, neither of how many landmarks are in the map. The g$^2$o framework does not provides a way to solve the correspondence problem, so a method must be developed to address de issue. In this work the method implemented is based in the one presented in~\cite{graphslam}, with some differences to take in account the speed,  false alarms\todo{review}. 

\subsection{The Correspondence Test}

The premise of the method is as follows: a correspondence test is developed to check the likelihood of two landmarks being the same. If the likelihood is high enough, the landmarks are merged into one. 

To justify mathematically the correspondence test the variable $\bs{\Delta}_{j,k}=\bs{m}_j-\bs{m}_k$ is defined as the difference in position of landmark $\bs{m}_j$ and $\bs{m}_k$. The posterior probability of $\bs{\Delta}_{j,k}$ over the measurements and control inputs is given by:

\begin{equation}
p(\bs{\Delta}_{j,k}|\bs{z}_{1:k},\bs{u}_{1:k})=
\det(2\pi \bs{\Sigma}_{\bs{\Delta}_{j,k}})^{-\frac{1}{2}}\exp\left\lbrace -\frac{1}{2}(\bs{\Delta}_{j,k}-\bs{\mu}_{\bs{\Delta}_{j,k}})^T
\bs{\Sigma}_{\bs{\Delta}_{j,k}}^{-1}(\bs{\Delta}_{j,k}-\bs{\mu}_{\bs{\Delta}_{j,k}})\right\rbrace
\label{eq:correspondence-test}
\end{equation}

Where $\bs{\mu}_{\bs{\Delta}_{j,k}}$ is the current estimate of the landmark's difference. It can be easily computed as $\bs{\mu}_{\bs{m}_j}-\bs{\mu}_{\bs{m}_k}$, where $\bs{\mu}_{\bs{m}_j}$ and $\bs{\mu}_{\bs{m}_k}$ are the current estimate of both landmarks.

Matrix $\bs{\Sigma}_{\bs{\Delta}_{j,k}}$ is the covariance matrix marginalized over landmarks $j$ and $k$. Since it has been assumed a normal distribution for the estimate, $\bs{\Sigma}_{\bs{\Delta}_{j,k}}$ can be computed using the marginalization lemma~\cite{graphslam}. In practice, g$^2$o provides a function to compute $\bs{\Sigma}_{\bs{\Delta}_{j,k}}$.

\todo[inline]{explain marginal cov equation?}

When two landmarks are equivalent it is expected that their position is the same, hence $\Delta_{j,k}=0$. Evaluating this in the posterior probability~\eqref{eq:correspondence-test} gives the likelihood of landmark equivalence:

\begin{equation}
\pi_{j=k} \defi
p(\bs{\Delta}_{j,k}=0|\bs{z}_{1:k},\bs{u}_{1:k})=
\det(2\pi\bs{\Sigma}_{\bs{\Delta}_{j,k}})^{-\frac{1}{2}}
\exp\left\lbrace-\frac{1}{2}\bs{\mu}_{\bs{\Delta}_{j,k}}^T\bs{\Sigma}_{\bs{\Delta}_{j,k}}^{-1}\bs{\mu}_{\bs{\Delta}_{j,k}}\right\rbrace
\end{equation}

Then the correspondence test consist in assert a landmark equivalence when the likelihood $\pi_{j=k}$ is grater that a user defined threshold $\chi$. Intuitively a greater threshold means being more strict in considering a landmark merging.

\subsection{The Unknown Correspondence Algorithm}

Once the correspondence test is defined, it can be used to implement an GraphSLAM algorithm with unknown data association. 

The algorithm works as follows: first all landmarks are initialized as if each measurement correspond to an individual landmark. The correspondence test is run over every possible pair of landmarks, merging landmarks who pass the test. After the tests are finished, the estimate is updated running the solver in the same way as in the case of known correspondence. Then the correspondence test is run again and the solver is run afterward. Correspondence test and solver are alternated until no more landmark associations are found. 

Algorithm~\ref{alg:unknown-correspondence} presents the unknown correspondence algorithm in pseudocode. 

\begin{algorithm}
    \caption{GraphSLAM Unknown Correspondence}
    \label{alg:unknown-correspondence}
    \begin{algorithmic}[1]
        \Require optimizer, data
        \State optimizer.setParameters(parameters)
        \State optimizer.loadData(data)
        \State optimizer.genInitialGuess()
        \State
        \While{association found} 
            \ForAll{pair of landmark ($j$,$k$)} 
                \If{correspondenceTest($j$,$k$) $\geq \chi$} 
                    \State optimizer.merge($j$,$k$) 
                \EndIf 
            \EndFor
            \State optimizer.solve(numberIterations)
        \EndWhile
        \State
        \State optimizer.writeData()
    \end{algorithmic}
\end{algorithm}

\subsection{Speeding up the Unknown Correspondence Algorithm}

Algorithm~\ref{alg:unknown-correspondence} is inefficient. In particular the correspondence test is run over every pair of landmarks at each iteration. The possible pairs is quadratic in the number of landmarks, furthermore, even landmarks that are obviously not equivalent, such as landmarks greatly separated, are tested. Empirical testing have shown that the bottleneck of the algorithm is the the correspondence test, in particular, the computation of the marginalized covariance $\bs{\Sigma}_{\bs{\Delta}_{j,k}}$, so is necessary call this function as less as possible. The optimization of the algorithm is essential to run GraphSLAM in large datasets. The next subsections presents the strategies adopted to improve the algorithm performance.

\subsubsection{Incremental Optimization}

\subsubsection{Distant Test}

\subsubsection{Late Landmark Merging}
\chapter{Results}
\label{chap:results}

In this chapter the results of the GraphSLAM algorithm are presented for different test scenarios. In section~\ref{sec:known-asso-res} the algorithm is tested for the simple case of known data association and simulated data. A parameters variation analysis is made, and their effect in the path estimation is shown. In section~\ref{sec:unknown-asso-res} a similar analysis is performed, but for the case of unknown data association. In this case, new parameters related to the data association algorithm are tested. Finally in section~\ref{sec:real-data-res}, the GraphSLAM algorithm is run in more realistic data. This data includes robot simulated with Gazebo\footnote{http://gazebosim.org/}, a much more realistic robot simulator, and data obtained from real robots in outdoor environments. 

Through this tests several parameters and methods must be chosen for the algorithm to work properly. To limit the scope of this work, the sparse solver and the optimization algorithm are fixed and used in all the following tests. CSpase library and the Cholesky decomposition is used for the sparse solver, and the Levenberg-Marquardt method is used as the optimization algorithm. Also whenever the robust kernel method is used, Huber is the chosen kernel function (see section~\ref{sec:known-asso-imp}). 

\section{Known Data Association}
\label{sec:known-asso-res}

The known data association case is the easiest one of the three scenarios, because correspondence between landmarks is given a priori. In this case the user must simply set the parameters and run the solver as stated in Algorithm~\ref{alg:known-correspondence}. 

The parameters to be set in this case are the following:

\begin{itemize}
    \item $n_p$: number of poses of the robot path.
    \item $n_l$: number of landmarks in the map.
    \item $i_{op}$: odometry position information (inverse of variance).
    \item $i_{oa}$: odometry angle information.
    \item $i_{lp}$: landmark position information.
    \item $it$: number of iterations for the optimization algorithm to stop.
    \item $k_w$: Width of the chosen robust kernel.
\end{itemize}

Where $n_p$, $n_l$, $i_{op}$, $i_{oa}$, and $i_{lp}$ are parameters regarding the robot behavior in the test. They are passed to the simulator. The parameters $it$ and $k_w$ define de optimization strategy.

Through all the test made in this work, it is assumed that the information matrix of odometry and measurements models (same as~\eqref{eq:info-matrices}) are diagonal, i.e., their variables are not correlated. Furthermore, it is assumed that these matrices have the following structure:

\begin{equation}
\bs{R}^{-1}_k = \begin{pmatrix}
i_{op} & 0 & 0\\
0 & i_{op} & 0\\
0 & 0 & i_{oa}
\end{pmatrix} \;\;
\bs{Q}^{-1}_k = \begin{pmatrix}
i_{lp} & 0\\
0 & i_{lp}
\end{pmatrix} 
\label{eq:info-matrices-tests}
\end{equation}

This means that the robot experiences same uncertainty in the $x$ and $y$ axis, for both, motion and measurement model. 

The first test, \nameref{sec:test-i}, is run with the parameters of Table~\ref{tab:test-i}. 

The simulation is constrained to a 2D world with $x\in[-25,25]$, $y\in[-15,15]$. At each step the simulator moves the robot 1 unit of distance and in turns randomly an angle of $\theta=0^\circ$, $90^\circ$, or $-90^\circ$. All simulations start from at $(0,0)$.

A kernel width of 1 unit is chosen heuristically, looking at the distance between landmarks. 

The results of the test are shown in Figure~\ref{fig:test-i}. In Figure~\ref{fig:test-ia} the initial guess of the solver is plotted on the left, and the posterior estimation after the solver on the right. The groundtruth is added in both graphs for comparison. Landmarks not observed by the robot are not shown.  It can be seen that the path of the initial guess rapidly diverges from the real solution. On the other hand, the solver estimation fits quite well with the groundtruth, both for the path and the landmarks.

To have a more quantitative view of the error made by the solver, the cumulative path error for every step is shown in Figure~\ref{fig:test-ib}.  The cumulative normalized error is given by:

\begin{equation}
error (i) = \frac{1}{i} \sum_{i} ||pos_{GT}-pos_{est}||
\end{equation} 

Where $i$ is the path's timestep. $pos_{GT}$ and $pos_{est}$ are the groundtruth and estimated position respectively. Then the error is the sum of the euclidean distance of both positions, normalized by the number of steps. 

It can be seen that the error start increasing early in the path, but then it get stabilized. This is the effect of aggregating the information of the measurements and running the optimizer. As long as the robot keep measuring the same landmarks, it can maintain its error low. 


\subsubsection{Test I.}
\label{sec:test-i}

\begin{table}[htbp!]
\centering
\begin{tabular}{|c|c|c|c|c|c|c|}
\hline
$n_p$ & $n_l$ & $i_{op}$ & $i_{oa}$ & $i_{lp}$ & $it$ & $k_w$\\
\hline \hline
300 & 40 & 1000 & 1000 & 1000 & 20 & 1\\
\hline 
\end{tabular}
\caption{Parameters for Test I.}
\label{tab:test-i}
\end{table}

\begin{figure}[htbp!]
\centering
\begin{subfigure}[b]{\estWidth\textwidth}
\includegraphics[width=\textwidth]{imagenes/tests/known/res_it_20_nl_40_op_1000_oa_1000_lp_1000_ds_300_kw_1.pdf}
\caption{Initial guess and solver estimation.}
\label{fig:test-ia}
\end{subfigure}\\
\begin{subfigure}[b]{\errorWidth\textwidth}
\includegraphics[width=\textwidth]{imagenes/tests/known/res_it_20_nl_40_op_1000_oa_1000_lp_1000_ds_300_kw_1_path.pdf}
\caption{Path normalized error.}
\label{fig:test-ib}
\end{subfigure}
\caption{Results for Test I.}
\label{fig:test-i}
\end{figure}
\clearpage

In the next test all the information parameters of the robot are decreased simultaneously. Is intended to show the effect in the estimation when adding uncertainty to the robot.

\nameref{sec:test-ii}, \nameref{sec:test-iii}, and \nameref{sec:test-iv}, shows the results when $i_{op}=i_{oa}=i_{lp}=100$, $i_{op}=i_{oa}=i_{lp}=10$, and $i_{op}=i_{oa}=i_{lp}=1$ respectively.

It can be seen that the algorithm can retrieve the path with relative success when the information is 100 and 10, even when the initial guess is very far off. Only when the information is 1, the algorithm breaks.

\subsubsection{Test II}
\label{sec:test-ii}

\begin{table}[htbp!]
    \centering
    \begin{tabular}{|c|c|c|c|c|c|c|}
        \hline
        $n_p$ & $n_l$ & $i_{op}$ & $i_{oa}$ & $i_{lp}$ & $it$ & $k_w$\\
        \hline \hline
        300 & 40 & 100 & 100 & 100 & 20 & 1\\
        \hline 
    \end{tabular}
    \caption{Parameters for Test II.}
    \label{tab:test-ii}
\end{table}

\begin{figure}[htbp!]
    \centering
    \begin{subfigure}[b]{\estWidth\textwidth}
        \includegraphics[width=\textwidth]{imagenes/tests/known/res_it_20_nl_40_op_100_oa_100_lp_100_ds_300_kw_1.pdf}
        \caption{Initial guess and solver estimation.}
        \label{fig:test-iia}
    \end{subfigure}\\
    \begin{subfigure}[b]{\errorWidth\textwidth}
        \includegraphics[width=\textwidth]{imagenes/tests/known/res_it_20_nl_40_op_100_oa_100_lp_100_ds_300_kw_1_path.pdf}
        \caption{Path normalized error.}
        \label{fig:test-iib}
    \end{subfigure}
    \caption{Results for Test II.}
    \label{fig:test-ii}
\end{figure}
\clearpage

\subsubsection{Test III}
\label{sec:test-iii}

\begin{table}[htbp!]
    \centering
    \begin{tabular}{|c|c|c|c|c|c|c|}
        \hline
        $n_p$ & $n_l$ & $i_{op}$ & $i_{oa}$ & $i_{lp}$ & $it$ & $k_w$\\
        \hline \hline
        300 & 40 & 10 & 10 & 10 & 20 & 1\\
        \hline 
    \end{tabular}
    \caption{Parameters for Test III.}
    \label{tab:test-iii}
\end{table}

\begin{figure}[htbp!]
    \centering
    \begin{subfigure}[b]{\estWidth\textwidth}
        \includegraphics[width=\textwidth]{imagenes/tests/known/res_it_20_nl_40_op_10_oa_10_lp_10_ds_300_kw_1.pdf}
        \caption{Initial guess and solver estimation.}
        \label{fig:test-iiia}
    \end{subfigure}\\
    \begin{subfigure}[b]{\errorWidth\textwidth}
        \includegraphics[width=\textwidth]{imagenes/tests/known/res_it_20_nl_40_op_10_oa_10_lp_10_ds_300_kw_1_path.pdf}
        \caption{Path normalized error.}
        \label{fig:test-iiib}
    \end{subfigure}
    \caption{Results for Test III.}
    \label{fig:test-iii}
\end{figure}
\clearpage

\subsubsection{Test IV}
\label{sec:test-iv}

\begin{table}[htbp!]
    \centering
    \begin{tabular}{|c|c|c|c|c|c|c|}
        \hline
        $n_p$ & $n_l$ & $i_{op}$ & $i_{oa}$ & $i_{lp}$ & $it$ & $k_w$\\
        \hline \hline
        300 & 40 & 1 & 1 & 1 & 20 & 1\\
        \hline 
    \end{tabular}
    \caption{Parameters for Test IV.}
    \label{tab:test-iv}
\end{table}

\begin{figure}[htbp!]
    \centering
    \begin{subfigure}[b]{\estWidth\textwidth}
        \includegraphics[width=\textwidth]{imagenes/tests/known/res_it_20_nl_40_op_1_oa_1_lp_1_ds_300_kw_1.pdf}
        \caption{Initial guess and solver estimation.}
        \label{fig:test-iva}
    \end{subfigure}\\
    \begin{subfigure}[b]{\errorWidth\textwidth}
        \includegraphics[width=\textwidth]{imagenes/tests/known/res_it_20_nl_40_op_1_oa_1_lp_1_ds_300_kw_1_path.pdf}
        \caption{Path normalized error.}
        \label{fig:test-ivb}
    \end{subfigure}
    \caption{Results for Test IV.}
    \label{fig:test-iv}
\end{figure}
\clearpage

\nameref{sec:test-v} shows the effect running the algorithm with a low number of iterations. In this case $it=1$. It can be seen that the algorithm was not able to converge properly, in contrast to~\nameref{sec:test-i}.

\subsubsection{Test V.}
\label{sec:test-v}

\begin{table}[htbp!]
    \centering
    \begin{tabular}{|c|c|c|c|c|c|c|}
        \hline
        $n_p$ & $n_l$ & $i_{op}$ & $i_{oa}$ & $i_{lp}$ & $it$ & $k_w$\\
        \hline \hline
        300 & 40 & 1000 & 1000 & 1000 & 1 & 1\\
        \hline 
    \end{tabular}
    \caption{Parameters for Test I.}
    \label{tab:test-v}
\end{table}

\begin{figure}[htbp!]
    \centering
    \begin{subfigure}[b]{\estWidth\textwidth}
        \includegraphics[width=\textwidth]{imagenes/tests/known/res_it_1_nl_40_op_1000_oa_1000_lp_1000_ds_300_kw_1.pdf}
        \caption{Initial guess and solver estimation.}
        \label{fig:test-va}
    \end{subfigure}\\
    \begin{subfigure}[b]{\errorWidth\textwidth}
        \includegraphics[width=\textwidth]{imagenes/tests/known/res_it_1_nl_40_op_1000_oa_1000_lp_1000_ds_300_kw_1_path.pdf}
        \caption{Path normalized error.}
        \label{fig:test-vb}
    \end{subfigure}
    \caption{Results for Test V.}
    \label{fig:test-v}
\end{figure}
\clearpage

In~\nameref{sec:test-vi} the kernel width was reduced to $kw=0.1$, meaning that the effect of the kernel is applied earlier in distance. It can be seen that the new width actually improves the results, getting a normalized error for the full path of around 0.08.

\subsubsection{Test VI.}
\label{sec:test-vi}

\begin{table}[htbp!]
    \centering
    \begin{tabular}{|c|c|c|c|c|c|c|}
        \hline
        $n_p$ & $n_l$ & $i_{op}$ & $i_{oa}$ & $i_{lp}$ & $it$ & $k_w$\\
        \hline \hline
        300 & 40 & 1000 & 1000 & 1000 & 20 & 0.1\\
        \hline 
    \end{tabular}
    \caption{Parameters for Test VI.}
    \label{tab:test-vi}
\end{table}

\begin{figure}[htbp!]
    \centering
    \begin{subfigure}[b]{\estWidth\textwidth}
        \includegraphics[width=\textwidth]{imagenes/tests/known/{res_it_20_nl_40_op_1000_oa_1000_lp_1000_ds_300_kw_0.1}.pdf}
        \caption{Initial guess and solver estimation.}
        \label{fig:test-via}
    \end{subfigure}\\
    \begin{subfigure}[b]{\errorWidth\textwidth}
        \includegraphics[width=\textwidth]{imagenes/tests/known/{res_it_20_nl_40_op_1000_oa_1000_lp_1000_ds_300_kw_0.1_path}.pdf}
        \caption{Path normalized error.}
        \label{fig:test-vib}
    \end{subfigure}
    \caption{Results for Test VI.}
    \label{fig:test-vi}
\end{figure}
\clearpage

In terms of speed the algorithm performs quite fast, taking around $1[s]$ for all the tests.

\section{Unknown Data Association}
\label{sec:unknown-asso-res}

In this section similar tests are made, but in this case it is assumed unknown data association of the landmarks. The same simulator is used to generate the data, and Algorithm~\ref{alg:final} to compute the estimate.

Along with the parameters used in the known correspondence case, new parameters are used in this tests, which are the followings:

\begin{itemize}
\item $\chi$: likelihood threshold for data association.
\item $dt$: maximum distance for distance test.
\item $io$: inter full optimization frequency.
\item $ps$: Pose skipping.
\end{itemize}

Where $\xi$ and $dt$ are the thresholds used in Algorithm~\ref{alg:incremental-data-association}, $io$ is the number of incremental optimization between two full optimizations, and $ps$ is the number of pose between optimizations. 

\nameref{sec:test-vii} show the results for a successful test with unknown data association. Notice that in the initial guess there are several more landmarks that in the groundtruth. That is because the initial guess consider every single measurement as an independent landmark. However in the result of the solver, the algorithm is able to associate the the measurements with the corresponding landmark, and correct the path of the robot. Nevertheless there are 4 cases that is unable to associate correctly.

The parameter $dt$ is set to $\infty$ so no distant test is performed. In table~\ref{tab:test-vii}, variable $t$ correspond to the total time of the algorithm. 

\subsubsection{Test VII.}
\label{sec:test-vii}

\begin{table}[htbp!]
    \centering
    \begin{tabular}{|c|c|c|c|c|c|c|c|c|c|c|c|}
        \hline
        $n_p$ & $n_l$ & $i_{op}$ & $i_{oa}$ & $i_{lp}$ & $it$ & $k_w$ & $\chi$ & $dt$ & $io$ & $ps$ & $t [s]$\\
        \hline \hline
        400 & 30 & 1000 & 10000 & 1000 & 20 & 1 & 0.1 & $\infty$ & 400 & 10 & 13\\
        \hline 
    \end{tabular}
    \caption{Parameters for Test VII.}
    \label{tab:test-vii}
\end{table}

\begin{figure}[htbp!]
    \centering
    \begin{subfigure}[b]{\estWidth\textwidth}
        \includegraphics[width=\textwidth]{imagenes/tests/unknown/{res_it_20_xi_0.1_nl_30_op_1000_oa_10000_lp_1000_dsk_1_io_400_ds_400_dt_0_kw_1_ps_10}.pdf}
        \caption{Initial guess and solver estimation.}
        \label{fig:test-viia}
    \end{subfigure}\\
    \begin{subfigure}[b]{\errorWidth\textwidth}
        \includegraphics[width=\textwidth]{imagenes/tests/unknown/{res_it_20_xi_0.1_nl_30_op_1000_oa_10000_lp_1000_dsk_1_io_400_ds_400_dt_0_kw_1_ps_10_path}.pdf}
        \caption{Path normalized error.}
        \label{fig:test-viib}
    \end{subfigure}
    \caption{Results for Test VII.}
    \label{fig:test-vii}
\end{figure}
\clearpage

In the next tests the threshold $\chi$ is modified. In~\nameref{sec:test-viii} the threshold is set ot a very low value $\chi=10^-100$. It can be seen that the algorithm merge non equivalent landmarks, which make the estimated path to degrade. In~\nameref{sec:test-ix} and \nameref{sec:test-x} $\chi$ is set to 1 and 3 respectively. In these cases the thresholds are so high that the algorithm is unable to associate equivalent landmarks. This cause that the estimated path gradually drift.

\subsubsection{Test VIII.}
\label{sec:test-viii}

\begin{table}[htbp!]
    \centering
    \begin{tabular}{|c|c|c|c|c|c|c|c|c|c|c|c|}
        \hline
        $n_p$ & $n_l$ & $i_{op}$ & $i_{oa}$ & $i_{lp}$ & $it$ & $k_w$ & $\chi$ & $dt$ & $io$ & $ps$ & $t [s]$\\
        \hline \hline
        400 & 30 & 1000 & 10000 & 1000 & 20 & 1 & $10^{-100}$ & $\infty$ & 400 & 10 & 94\\
        \hline 
    \end{tabular}
    \caption{Parameters for Test VIII.}
    \label{tab:test-viii}
\end{table}

\begin{figure}[htbp!]
    \centering
    \begin{subfigure}[b]{\estWidth\textwidth}
        \includegraphics[width=\textwidth]{imagenes/tests/unknown/{res_it_20_xi_1e-100_nl_30_op_1000_oa_10000_lp_1000_dsk_1_io_400_ds_400_dt_0_kw_1_ps_10}.pdf}
        \caption{Initial guess and solver estimation.}
        \label{fig:test-viiia}
    \end{subfigure}\\
    \begin{subfigure}[b]{\errorWidth\textwidth}
        \includegraphics[width=\textwidth]{imagenes/tests/unknown/{res_it_20_xi_1e-100_nl_30_op_1000_oa_10000_lp_1000_dsk_1_io_400_ds_400_dt_0_kw_1_ps_10_path}.pdf}
        \caption{Path normalized error.}
        \label{fig:test-viiib}
    \end{subfigure}
    \caption{Results for Test VIII.}
    \label{fig:test-viii}
\end{figure}
\clearpage

\subsubsection{Test IX.}
\label{sec:test-ix}

\begin{table}[htbp!]
    \centering
    \begin{tabular}{|c|c|c|c|c|c|c|c|c|c|c|c|}
        \hline
        $n_p$ & $n_l$ & $i_{op}$ & $i_{oa}$ & $i_{lp}$ & $it$ & $k_w$ & $\chi$ & $dt$ & $io$ & $ps$ & $t [s]$\\
        \hline \hline
        400 & 30 & 1000 & 10000 & 1000 & 20 & 1 & 1 & $\infty$ & 400 & 10 & 10\\
        \hline 
    \end{tabular}
    \caption{Parameters for Test IX.}
    \label{tab:test-ix}
\end{table}

\begin{figure}[htbp!]
    \centering
    \begin{subfigure}[b]{\estWidth\textwidth}
        \includegraphics[width=\textwidth]{imagenes/tests/unknown/{res_it_20_xi_1_nl_30_op_1000_oa_10000_lp_1000_dsk_1_io_400_ds_400_dt_0_kw_1_ps_10}.pdf}
        \caption{Initial guess and solver estimation.}
        \label{fig:test-ixa}
    \end{subfigure}\\
    \begin{subfigure}[b]{\errorWidth\textwidth}
        \includegraphics[width=\textwidth]{imagenes/tests/unknown/{res_it_20_xi_1_nl_30_op_1000_oa_10000_lp_1000_dsk_1_io_400_ds_400_dt_0_kw_1_ps_10_path}.pdf}
        \caption{Path normalized error.}
        \label{fig:test-ixb}
    \end{subfigure}
    \caption{Results for Test IX.}
    \label{fig:test-ix}
\end{figure}
\clearpage

\subsubsection{Test X.}
\label{sec:test-x}

\begin{table}[htbp!]
    \centering
    \begin{tabular}{|c|c|c|c|c|c|c|c|c|c|c|c|}
        \hline
        $n_p$ & $n_l$ & $i_{op}$ & $i_{oa}$ & $i_{lp}$ & $it$ & $k_w$ & $\chi$ & $dt$ & $io$ & $ps$ & $t [s]$\\
        \hline \hline
        400 & 30 & 1000 & 10000 & 1000 & 20 & 1 & 3 & $\infty$ & 400 & 10 & 17\\
        \hline 
    \end{tabular}
    \caption{Parameters for Test X.}
    \label{tab:test-x}
\end{table}

\begin{figure}[htbp!]
    \centering
    \begin{subfigure}[b]{\estWidth\textwidth}
        \includegraphics[width=\textwidth]{imagenes/tests/unknown/{res_it_20_xi_3_nl_30_op_1000_oa_10000_lp_1000_dsk_1_io_400_ds_400_dt_0_kw_1_ps_10}.pdf}
        \caption{Initial guess and solver estimation.}
        \label{fig:test-xa}
    \end{subfigure}\\
    \begin{subfigure}[b]{\errorWidth\textwidth}
        \includegraphics[width=\textwidth]{imagenes/tests/unknown/{res_it_20_xi_3_nl_30_op_1000_oa_10000_lp_1000_dsk_1_io_400_ds_400_dt_0_kw_1_ps_10_path}.pdf}
        \caption{Path normalized error.}
        \label{fig:test-xb}
    \end{subfigure}
    \caption{Results for Test X.}
    \label{fig:test-x}
\end{figure}
\clearpage

In the next test $\chi$ gets the value of $\infty$ and $dt=1$, so that in this case the algorithm according to distance instead of likelihood. Every landmark at a distance of 1 unit will be automatically associated. Looking at Figure~\ref{fig:test-xia} it can be seen that now all landmarks are correctly associated. However Figure~\ref{fig:test-xib} shows that the estimation has a greater cumulative error. Also, the distant test is faster than the correspondence test, taking only $5[s]$. Therefore a threshold exists between using the distant test and the correspondence test as a mean to solve unknown association.

\subsubsection{Test XI.}
\label{sec:test-xi}

\begin{table}[htbp!]
    \centering
    \begin{tabular}{|c|c|c|c|c|c|c|c|c|c|c|c|}
        \hline
        $n_p$ & $n_l$ & $i_{op}$ & $i_{oa}$ & $i_{lp}$ & $it$ & $k_w$ & $\chi$ & $dt$ & $io$ & $ps$ & $t [s]$\\
        \hline \hline
        400 & 30 & 1000 & 10000 & 1000 & 20 & 1 & $\infty$ & 1 & 400 & 10 & 5\\
        \hline 
    \end{tabular}
    \caption{Parameters for Test XI.}
    \label{tab:test-xi}
\end{table}

\begin{figure}[htbp!]
    \centering
    \begin{subfigure}[b]{\estWidth\textwidth}
        \includegraphics[width=\textwidth]{imagenes/tests/unknown/{res_it_20_xi_0_nl_30_op_1000_oa_10000_lp_1000_dsk_1_io_400_ds_400_dt_1_kw_1_ps_10}.pdf}
        \caption{Initial guess and solver estimation.}
        \label{fig:test-xia}
    \end{subfigure}\\
    \begin{subfigure}[b]{\errorWidth\textwidth}
        \includegraphics[width=\textwidth]{imagenes/tests/unknown/{res_it_20_xi_0_nl_30_op_1000_oa_10000_lp_1000_dsk_1_io_400_ds_400_dt_1_kw_1_ps_10_path}.pdf}
        \caption{Path normalized error.}
        \label{fig:test-xib}
    \end{subfigure}
    \caption{Results for Test XI.}
    \label{fig:test-xi}
\end{figure}
\clearpage

In~\nameref{sec:test-xii} the pose skip parameter is increase to $ps=100$. the idea is to do as least optimization as possible to increase the algorithm speed. It can be seen however that, due to the lack of optimization, the algorithm is unable to correct the path correctly and to associate the landmarks. This errors also cause that the algorithm actually takes longer than in~\nameref{sec:test-vii}. 

\subsubsection{Test XII.}
\label{sec:test-xii}

\begin{table}[htbp!]
    \centering
    \begin{tabular}{|c|c|c|c|c|c|c|c|c|c|c|c|}
        \hline
        $n_p$ & $n_l$ & $i_{op}$ & $i_{oa}$ & $i_{lp}$ & $it$ & $k_w$ & $\chi$ & $dt$ & $io$ & $ps$ & $t [s]$\\
        \hline \hline
        400 & 30 & 1000 & 10000 & 1000 & 20 & 1 & 0.1 & $\infty$ & 400 & 100 & 21\\
        \hline 
    \end{tabular}
    \caption{Parameters for Test XII.}
    \label{tab:test-xii}
\end{table}

\begin{figure}[htbp!]
    \centering
    \begin{subfigure}[b]{\estWidth\textwidth}
        \includegraphics[width=\textwidth]{imagenes/tests/unknown/{res_it_20_xi_0.1_nl_30_op_1000_oa_10000_lp_1000_dsk_1_io_400_ds_400_dt_0_kw_1_ps_100}.pdf}
        \caption{Initial guess and solver estimation.}
        \label{fig:test-xiia}
    \end{subfigure}\\
    \begin{subfigure}[b]{\errorWidth\textwidth}
        \includegraphics[width=\textwidth]{imagenes/tests/unknown/{res_it_20_xi_0.1_nl_30_op_1000_oa_10000_lp_1000_dsk_1_io_400_ds_400_dt_0_kw_1_ps_100_path}.pdf}
        \caption{Path normalized error.}
        \label{fig:test-xiib}
    \end{subfigure}
    \caption{Results for Test XII.}
    \label{fig:test-xii}
\end{figure}
\clearpage
 
\section{Real Data}
\label{sec:real-data-res}
\begin{conclusion}
\label{chap:conclusion}

\end{conclusion}

%%\newglossaryentry{k}
%{
%	name={\ensuremath{k}},
%	description={discrete time variable},
%	sort=k
%}

%\newglossaryentry{u_k}
%{
%	name={\ensuremath{\bs{u}_k}},
%	description={vector of control inputs at time $k$},
%	sort=uk
%}

\newacronym{slam}{SLAM}{Simultaneous Localization And Mapping}
\newacronym{pdf}{PDF}{Probability Density Function} % opcional

\bibliographystyle{plain}
\bibliography{bibliografia}

% \input{anexo_apendices.tex} % opcionales

\end{document}
