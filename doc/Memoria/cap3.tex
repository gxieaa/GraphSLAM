\chapter{Methodology and Implementation}
\label{chap:implementation}
 
In this chapter the implementation of the GraphSLAM algorithm is presented and explained. The implementation created is capable of solving the SLAM problem in an offline manner for a 2D scenario, in the cases of known and unknown data association.

The g$^2$o framework used in this work provides of a least squares solver for the optimization of equation~\eqref{eq:minimization}, as well as a protocol to define a graph in the SLAM problem. g$^2$o is well optimized and it has several options for the solver, so the known correspondence version of the GraphSLAM algorithm is relative straightforward to implement.

However, g$^2$o doesn't provide a way to handle unknown data association, so the main goal of this work is to implement a method for solving the correspondence problem in an efficient manner.

\section{The Known Correspondence Case and the g$^2$o Framework}

The first step to implementing the GraphSLAM algorithm is to define a protocol to store and interpret the data on a graph. g$^2$o already provides such protocol. The data to be stored comes in two types: data from nodes and data from edges. In SLAM context, nodes itself can be of two types, pose nodes and landmark nodes. Pose nodes represent the pose of the robot, and in the 2D case they consists in 3 variables: robot's $x$ and $y$ position, and robot's bearing $\theta$. In g$^2$o a pose node is denotedd with the keyword \texttt{VERTEX_SE2}\footnote{Vertex is synonimus of node. SE2 is the non-euclidian space of two spatial dimensions and a angular dimension.}. Landmark nodes represent the 2D ($x$ and $y$) position of a landmark. They are denoted with the keyword \texttt{VERTEX_XY}.




\section{The Unknown Correspondence Case}
