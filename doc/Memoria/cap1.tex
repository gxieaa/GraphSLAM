\chapter{Introduction}

This work is immerse in the context of Mobile Robotics. Mobile
Robotics is the branch of engineering that study machine that can move in an
environment, that is, they can change their location over time. Traditionally,
robots were usually used to do a simple, repetitive task in a fixed location, such
as robots found in assembly lines. In contrast, mobile robots are more versatile
and able to do a wider variety of tasks, but at the cost of needing more complex
models to study and control them.

Mobile robots frequently have to work in unknown environments, with high
uncertainly. Researches have found that a good way to deal with this uncertainly
is to treat the involved variables in the problem as random variables. This way
the field of Probabilistic Robotics was born.

Two of the main problems to be solved in mobile robotics are localization and
mapping. Localization means to find an estimate of the location of a robot that is
moving on a scene. Mapping is the problem of constructing a map of an unknown
environment. When the agent in charge of constructing the map is a robot moving
in the same environment, both localization and mapping must be solve at the same
time. In Robotics this problem is called Simultaneous Localization and Mapping
(SLAM for short). SLAM is a widely studied problem in the academy, and wide
variety of solutions exists, all with their own advantages and disadvantages.
One particular solution for solving SLAM is GraphSLAM. GraphSLAM was
first developed by Sebastian Thrun and Michael Montemerlo in 2006, and to date
is considered one of the most robust solutions, in addition of been simple and
of relative low complexity. GraphSLAM represents the necessary information re-
garding the robot and the map as nodes of a graph. This provides the algorithm
with advantages, such as the ability to store the information efficiently, in sparse
matrices.

The main objective of this work is to implement the GraphSLAM
algorithm. To the author knowledge here is no freely available source code of
GraphSLAM, and its implementation is invaluable as a benchmark comparison for
newer SLAM algorithms.

The contribution of this work is to provide a fully functional SLAM
algorithm, which could be used for the navigation of robots in real world scenarios,
and for the realization of comparative analysis with other SLAM algorithms.

\section{General Objectives}

\section{Specific Objectives}

\section{Document Structure}
